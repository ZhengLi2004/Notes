\documentclass[a4paper,12pt,twoside,openany]{book}
\usepackage{amd}
\usepackage{mathrsfs}

% %--------------------------------------------------------------------------
% %         General Setting
% %--------------------------------------------------------------------------

\graphicspath{{Images/}{../Images/}} %Path of figures
\setkeys{Gin}{width=0.85\textwidth} %Size of figures
\setlength{\cftbeforechapskip}{3pt} %space between items in toc
\setlength{\parindent}{0.5cm} % Idk
\input{theorems.tex} % Theorems styles and colors
\usepackage[english]{babel} %Language

\setlist[itemize]{itemsep=3pt} % Adjust the length as needed
\setlist[enumerate]{itemsep=3pt} % Adjust the length as needed

\usepackage{lmodern} %  Latin Modern font
% \usepackage{newtxtext,newtxmath}




% %--------------------------------------------------------------------------
% %         General Informations
% %--------------------------------------------------------------------------
\newcommand{\BigTitle}{
    Topology
    }

\newcommand{\LittleTitle}{
    Written by James R. Munkres
    }

    
\begin{document}

% %--------------------------------------------------------------------------
% %         First pages 
% %--------------------------------------------------------------------------
\newgeometry{top=8cm,bottom=.5in,left=2cm,right=2cm}
\subfile{files/0.0.0.titlepage}
\restoregeometry
\thispagestyle{empty}
\setcounter{page}{0}
\tableofcontents
\thispagestyle{empty}
\setcounter{page}{0}

% %--------------------------------------------------------------------------
% %         Core of the document 
% %--------------------------------------------------------------------------

\chapter{Set Theory and Logic}

We shall assume that what is meant by a \textit{set} of objects is intuitively clear, and we shall proceed on that basis without analyzing the concept further.

\section{Fundamental Concepts}
\subsection{Basic Notation}

Commonly we shall use capital letters $A,B,\ldots$ to denote sets, an lowercase letters $a,b,\ldots$ to denote the \textit{\textbf{objects}} or \textit{\textbf{elements}} belonging to these sets.

We say that $A$ is a \textit{\textbf{subset}} of $B$ is every element of $A$ is also and element of $B$; and we express this fact by writing $$A\subset B.$$ If $A\subset B$ and $A$ is different from $B$, we say that $A$ is a \textit{\textbf{proper subset}} if $B$, and we write $$A\subsetneq B.$$ The relations $\subset$ and $\subsetneq$ are called \textit{\textbf{inclusion}} and \textit{\textbf{proper inclusion}}, respectively. If $A\subset B$, we also write $B\supset A$, which is read "$B$ \textit{\textbf{contains}} $A$."

\subsection{The Union of Sets and the Meaning of "or"}

Given two sets $A$ and $B$, one can form a set from them that consists of all the elements of $A$ together with all the elements of $B$. This set is called the \textit{\textbf{union}} of $A$ and $B$ and is denoted by $A\cup B$.

\subsection{The Intersection of Sets, the Empty Set, and the Meaning of "If ... Then"}

Given sets $A$ and $B$, another way one can form a set is to take the common part of $A$ and $B$. This set is called the \textit{\textbf{intersection}} of $A$ and $B$ and is denoted by $A\cap B$.

We introduce a special set that we call the \textit{\textbf{empty set}}, denoted by $\varnothing$, which we think of as "the set having no elements."

Using this convention, we express the statement that $A$ and $B$ have no elements in common by the equation $$A\cap B=\varnothing.$$ We also express this fact by saying that $A$ and $B$ are \textit{\textbf{disjoint}}.

Mathematicians have agreed always to use "if ... then" in the first sense, so that a statement of the form "If $P$, then $Q$" means that if $P$ is true, $Q$ is true also, but if $P$ is false, $Q$ may be either true or false.

As an example, consider the following statement about real numbers: $$If\;x>0,\;then\;x^3\ne 0.$$ It is a statement of the form, "If $P$, then $Q$," where $P$ is the phrase "$x>0$" (called the \textit{\textbf{hypothesis}} of the statement) and $Q$ is the phrase "$x^3\ne 0$" (called the \textit{\textbf{conclusion}} of the statement).

Another true statement about real numbers is the following: $$If\;x^2<0,\;then\;x=23;$$ in every case for which the hypothesis holds, the conclusion holds as well. Of course, it happens in this example that there are no cases for which the hypothesis holds. A statement of this sort is sometimes said to be \textit{\textbf{vacuously true}}.

\subsection{Contrapositive and Converse}

Give a statement of the form "If $P$, then $Q$," its \textit{\textbf{contrapositive}} is defined to be the statement "If $Q$ is not true, then $P$ is not true."

There is another statement that can be formed from the statement $P\Rightarrow Q$. It is the statement $$Q\Longrightarrow P,$$ which is called the \textit{\textbf{converse}} of $P\Rightarrow Q$.

\subsection{Negation}

If one wishes to form the contrapositive of the statement $P\Rightarrow Q$, one has to know how to form the statement "not $P$", which is called the \textit{\textbf{negation}} of $P$.

\subsection{The Difference of Two Sets}

There is one other operation on sets that is occasionally useful. It is the \textit{\textbf{difference}} of two sets, denoted by $A-B$, and defined as the set conssiting of those elements of $A$ that are not in $B$. It is sometimes called the \textit{\textbf{complement}} of $B$ relative to $A$, or the complement of $B$ \textit{in} $A$.

\subsection{Collections of Sets}

Given a set $A$, we can consider sets whose elements are subsets of $A$. In particular, we can consider the set of all subsets of $A$. This set is sometimes denoted by the symbol $\mathcal{P}(A)$ and is called the \textit{\textbf{power set}} of $A$.

When we have a set whose elements are sets, we shall often refer to it as a \textit{\textbf{collection}} of sets and denote it by a script letter.

\subsection{Arbitrary Unions and Intersections}

Given a collection $\mathcal{A}$ of sets, the \textit{\textbf{union}} of the elements of $\mathcal{A}$ is defined by the equation $$\bigcup_{A\in \mathcal{A}}A=\{x\mid x\in A\text{ for at least one }A\in\mathcal{A}\}.$$ The \textit{\textbf{intersection}} of the element of $\mathcal{A}$ is defined by the equation $$\bigcap_{A\in\mathcal{A}}A=\{x\mid x\in A\text{ for every }A\in\mathcal{A}\}.$$

\section{Functions}

First, we define the following:
\defn{}{
    A \textit{\textbf{rule of assignment}} is a subset $r$ of the cartesian product $C\times D$ of two sets, having the property that each element of $C$ appears as the first coordinate of \textit{at most one} ordered pair belonging to $r$.
}

Given a rule of assignment $r$, the \textit{\textbf{domain}} of $r$ is defined to be the subset of $C$ consisting of all first coordinates of elements of $r$, and the \textit{\textbf{image set}} of $r$ is defined as the subset of $D$ consisting of all second coordinates of elements of $r$.

Now we can say what a function is:
\defn{}{
    A \textit{\textbf{function}} $f$ is a rule of assignment $r$, together with a set $B$ that contains the image set of $r$. The domain $A$ of the rule $r$ is also called the \textit{\textbf{domain}} of the function $f$; the image set of $r$ is also called the \textit{\textbf{image set}} of $f$; and the set $B$ is called the \textit{\textbf{range}} of $f$.
}

If $f:A\rightarrow B$ and if $a$ is an element of $A$, we denote by $f(a)$ the unique element of $B$ that the rule determining $f$ assigns to $a$; it is called the \textit{\textbf{value}} of $f$ at $a$, or sometimes the \textit{\textbf{image}} of $a$ under $f$.

\defn{}{
    If $f:A\rightarrow B$ and if $A_0$ is a subset of $A$, we define the \textit{\textbf{restriction}} of $f$ to $A_0$ to be the function mapping $A_0$ into $B$ whose rule is $$\{(a,f(a))\mid a\in A_0\}.$$
}

\defn{}{
    Given functions $f:A\rightarrow B$ and $g:B\rightarrow C$, we define the \textit{\textbf{composite}} $g\circ f$ of $f$ and $g$ as the function $g\circ f:A\rightarrow C$ defined by the equation $(g\circ f)(a)=g(f(a))$.
}

\defn{}{
    A function $f:A\rightarrow B$ is said to be \textit{\textbf{injective}} (or \textit{\textbf{one-to-one}}) if for each pair of distinct points of $A$, their images under $f$ are distinct. It is said to be \textit{\textbf{surjective}} (or $f$ is said to map $A$ \textit{\textbf{onto}} $B$) if every element of $B$ is the image of some element of $A$ under the function $f$. If $f$ is both injective and surjective, it is said to be \textit{\textbf{bijective}} (or is called a \textit{\textbf{one-to-one correspondence}}).
}

If $f$ is bijective, there exists a function from $B$ to $A$ called the \textit{\textbf{inverse}} of $f$.

A useful criterion for showing that a given function $f$ is bijective is the following,

\lem{}{
    Let $f:A\rightarrow B$. If there are functions $g:B\rightarrow A$ and $h:B\rightarrow A$ such that $g(f(a))=a$ for every $a$ in $A$ and $f(h(b))=b$ for every $b$ in $B$, then $f$ is bijective and $g=h=f^{-1}$.
}

\defn{}{
    Let $f:A\rightarrow B$. If $A_0$ is a subset of $A$, we denote by $f(A_0)$ the set of all images of points of $A_0$ under the function $f$; this set is called the \textit{\textbf{image}} of $A_0$ under $f$. Formally, $$f(A_0)=\{b\mid b=f(a)\text{ for at least one }a\in A_0\}.$$ On the other hand, if $B_0$ is a subset of $B$, we denote by $f^{-1}(B_0)$ the set of all elements of $A$ whose images under $f$ lie in $B_0$; it is called the \textit{\textbf{preimage}} of $B_0$ under $f$ (or the "counterimage," or the "inverse image," of $B_0$).
}

\section{Relations}

\defn{}{
    A \textbf{\textit{relation}} on a set $A$ is a subset $C$ of the cartesian product $A\times A$.
}

\subsection{Equivalence Relations and Partitions}

An \textbf{\textit{equivalence relation}} on a set $A$ is a relation $C$ on $A$ having the following three properties:
\begin{enumerate}
    \item (Reflexivity) $xCx$ for every $x$ in $A$.
    \item (Symmetry) If $xCy$, then $yCx$.
    \item (Transitivity) If $xCy$ and $yCz$, then $xCz$.
\end{enumerate}

Given an equivalence relation $\sim$ on a set $A$ and an element $x$ of $A$, we define a certain subset $E$ of $A$, called the \textit{\textbf{equivalence class}} determined by $x$, by the equation $$E=\{y\mid y\sim x\}.$$ Equivalence classes have the following property:
\lem{}{
    Two equivalence classes $E$ and $E'$ are either disjoint or equal.
}

Given an equivalence relation on a set $A$, let us denote $\mathcal{E}$ the collection of all the equivalence classes determined by this relation. The collection $\mathcal{E}$ is a particular example of what is called a partition of $A$:
\defn{}{
    A \textit{\textbf{partition}} of a set $A$ is a collection of disjoint nonempty subsets of $A$ whose union is all of $A$.
}

\subsection{Order Relations}

A relation $C$ on a set $A$ is called an \textit{\textbf{order relation}} (or a \textit{\textbf{simple order}, or a \textit{\textbf{linear order}}}) if it has the following properties:
\begin{enumerate}
    \item (Comparability) For every $x$ and $y$ in $A$ for which $x\ne y$, either $xCy$ or $yCx$.
    \item (Nonreflexivity) For no $x$ in $A$ does the relation $xCx$ hold.
    \item (Transitivity) If $xCy$ and $yCz$, then $xCz$.
\end{enumerate}

\defn{}{
    If $X$ is a set and $<$ is an order relation on $X$, and if $a<b$, we use the notation $(a,b)$ to denote the set $$\{x\mid a<x<b\};$$ it is called an \textit{\textbf{open interval}} in $X$. If this set is empty, we call $a$ the \textit{\textbf{immediate predecessor}} of $b$, and we call $b$ the \textit{\textbf{immediate successor}} of $a$.
}

\defn{}{
    Suppose that $A$ and $B$ are two sets with order relations $<_A$ and $<_B$ respectively. We say that $A$ and $B$ have the same \textit{\textbf{order type}} if there is a bijective correspondence between them that preserves order; that is, if there exists a bijective function $f:A\rightarrow B$ such that $$a_1<_Aa_2\Longrightarrow f(a_1)<_Bf(a_2)$$
}

One interesting way of defining an order relation, which will be useful to us later in dealing with some examples, is the following:
\defn{}{
    Suppose that $A$ and $B$ are two sets with order relations $<_A$ and $<_B$ respectively. Define an order relation $<$ on $A\times B$ by defining $$a_1\times b_1<a_2\times b_2$$ if $a_1<_Aa_2$, or if $a_1=a_2$ and $b_1<_Bb_2$. It is called the \textit{\textbf{dictionary order relation}} on $A\times B$.
}

% %--------------------------------------------------------------------------
% %         Bibliographie 
% %--------------------------------------------------------------------------
\end{document}
