\documentclass[a4paper,12pt,twoside,openany]{book}
\usepackage{amd}

% %--------------------------------------------------------------------------
% %         General Setting
% %--------------------------------------------------------------------------

\graphicspath{{Images/}{../Images/}} %Path of figures
\setkeys{Gin}{width=0.85\textwidth} %Size of figures
\setlength{\cftbeforechapskip}{3pt} %space between items in toc
\setlength{\parindent}{0.5cm} % Idk
\input{theorems.tex} % Theorems styles and colors
\usepackage[english]{babel} %Language

\setlist[itemize]{itemsep=3pt} % Adjust the length as needed
\setlist[enumerate]{itemsep=3pt} % Adjust the length as needed

\usepackage{lmodern} %  Latin Modern font

% %--------------------------------------------------------------------------
% %         General Informations
% %--------------------------------------------------------------------------
\newcommand{\BigTitle}{
    Computer Organization and Design: The Hardware/Software Interface (RISC-V Edition)
    }

\newcommand{\LittleTitle}{
    Written by David A. Patterson et all
    }

\begin{document}

% %--------------------------------------------------------------------------
% %         First pages 
% %--------------------------------------------------------------------------
\newgeometry{top=8cm,bottom=.5in,left=2cm,right=2cm}
\subfile{files/0.0.0.titlepage}
\restoregeometry
\thispagestyle{empty}
\setcounter{page}{0}
\tableofcontents
\thispagestyle{empty}
\setcounter{page}{0}

% %--------------------------------------------------------------------------
% %         Core of the document 
% %--------------------------------------------------------------------------

\chapter{Computer Abstraction and Technology}
\section{Introduction}
\subsection{Traditional Classes of Computing Applications and Their Characteristics}

\textbf{Personal computers (PCs)} are possibly the best-known form of computing.

\textbf{Servers} are the modern form of what were once much larger computers, and are usually accessed only via a network.

Servers span the widest range in cost and capability. At the low end, a server may be little more than a desktop computer without a screen or keyboard and cost a thousand dollars. At the other extreme are \textbf{supercomputers}, which at the present consist of tens of thousands of processors and many \textbf{terabytes} of memory, and cost tens to hundreds of millions of dollars.

\defn{terabyte (TB)}{
    To reduce confusion, we now use the term \textbf{tebibyte (TiB)} for $2^{40}$ bytes, defining \textit{terabyte} (TB) to mean $10^{12}$ bytes.
}

\textbf{Embedded computers} are the largest class of computers and span the widest range of applications and performance.

\defn{Elaboration}{
    Many embedded processors are designed using \textit{processor cores}, a version of a processor written in a hardware description language.
}

\subsubsection{Welcome to the Post-PC Era}

Replacing the PC is the \textbf{personal mobile device (PMD)}.

Taking over from the conventional server is \textbf{Cloud Computing}, which relies upon giant datacenters that are now known as \textit{Warehouse Scale Computers} (WSCs). Indeed, \textbf{Software as a Service (SaaS)} deployed via the Cloud is revolutionizing the software industry just as PMDs and WSCs are revolutionizing the hardware industry.

\subsubsection{What You Can Learn in This Book}

% %--------------------------------------------------------------------------
% %         Bibliographie 
% %--------------------------------------------------------------------------
\end{document}
