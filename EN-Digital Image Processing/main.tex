\documentclass[a4paper,10pt,twoside]{book}
\usepackage{amd}

% %--------------------------------------------------------------------------
% %         General Setting
% %--------------------------------------------------------------------------

\graphicspath{{Images/}{../Images/}} %Path of figures
\setkeys{Gin}{width=0.85\textwidth} %Size of figures
\setlength{\cftbeforechapskip}{3pt} %space between items in toc
\setlength{\parindent}{0.5cm} % Idk
\input{theorems.tex} % Theorems styles and colors
\usepackage[english]{babel} %Language

\setlist[itemize]{itemsep=5pt} % Adjust the length as needed
\setlist[enumerate]{itemsep=5pt} % Adjust the length as needed



% \usepackage{lmodern} %  Latin Modern font
% \usepackage{newtxtext,newtxmath}




% %--------------------------------------------------------------------------
% %         General Informations
% %--------------------------------------------------------------------------
\newcommand{\BigTitle}{
    Digital Image Processing
    }

\newcommand{\LittleTitle}{
    By Rafael C. Gonzalez et all
    }

    
\begin{document}

% %--------------------------------------------------------------------------
% %         First pages 
% %--------------------------------------------------------------------------
\newgeometry{top=8cm,bottom=.5in,left=2cm,right=2cm}
\subfile{files/0.0.0.titlepage}
\restoregeometry
\thispagestyle{empty}
\setcounter{page}{0}
\tableofcontents
\thispagestyle{empty}
\setcounter{page}{0}

% %--------------------------------------------------------------------------
% %         Core of the document 
% %--------------------------------------------------------------------------

\chapter{Introduction}
\section{What Is Digital Image Processing?}

An image may be defined as a two-dimensional function, $f(x,y)$, where $x$ and $y$ are \textit{spatial} (plane) coordinates, and the amplitude of $f$ at any pair of coordinates $(x,y)$ is called the \textit{intensity} or \textit{gray level} of the image at that point. When $x$, $y$, and the intensity values of $f$ are all finite, discrete quantities, we call the image a \textit{digital image}. The field of \textit{digital image processing} refers to processing digital images by means of a digital computer. Note that a digital image is composed of a finite number of elements, each of which has a particular location and value. These elements are called \textit{picture elements}, \textit{image elements}, \textit{pels}, and \textit{pixels}. \textit{Pixel} is the term used most widely to denote the elements of a digital image.

\section{The Origins of Digital Image Processing}

The idea of a computer goes back to the invention of the abacus in Asia Minor, more than 5000 years ago. More recently, there were developments in the past two centuries that are the foundation of what we call a computer today. However, the basis for what we call a \textit{modern} digital computer dates back to only the 1940s with the introduction by John von Neumann of two key concepts: (1) a memory to hold a stored program and data, and (2) conditional branching.

Work on using computer techniques for improving images from a space probe began at the Jet Propulsion Laboratory (Pasadena, California) in 1964 when pictures of the moon transmitted by \textit{Ranger 7} were processed by a computer to correct various types of image distortion inherent in the on-board television camera. Figure \ref{Figure:1.4} shows the first image of the moon taken by \textit{Ranger 7} on July 31, 1964 at 9:09 {\footnotesize A.M.} Eastern Daylight Time (EDT), about 17 minutes before impacting the lunar surface (the markers, called \textit{reseau} marks, are used for geometric corrections). The imaging lessons learned with \textit{Ranger 7} served as the basis for improved methods used to enhance and restore images from the Surveyor missions to the moon, the Mariner series of flyby missions to Mars, the Apollo manned flights to the moon, and others.

\begin{figure}[htbp]
    \centering
    \includegraphics[width=\linewidth]{Figure1 4.png}
    \caption{The first picture of the moon by a U.S. spacecraft. \textit{Ranger 7} took this image on July 31, 1964 at 9:09 {\footnotesize A.M.} EDT, about 17 minutes before impacting the lunar surface. (Courtesy of NASA.)}
    \label{Figure:1.4}
\end{figure}

\section{Examples of Fields that Use Digital Image Processing}

Electromagnetic waves can be conceptualized as propagating sinusoidal waves of varying wavelengths, or they can be thought of as a stream of massless particles, each traveling in a wavelike pattern and moving at the speed of light. Each massless particle contains a certain amount (or bundle) of energy. Each bundle of energy is called a \textit{photon}.

\subsection{X-Ray Imaging}

The best known use of X-rays is medical diagnostics, but they also are used extensively in industry and other areas.

Angiography is another major application in an area called contrast-enhancement radiography. This procedure is used to obtain images (called \textit{angiograms}) of blood vessels.

\subsection{Imaging in the Visible and Infrared Bands}

\begin{table}[htbp]
    \centering
    \begin{tabular}{|clcl|}
        \hline
        \textbf{Band No.}&\multicolumn{1}{c}{\textbf{Name}}&\textbf{Wavelength ($\mu$m)}&\multicolumn{1}{c}{\textbf{Characteristics and Uses}}\vline\\ \hline
        1&Visible blue&0.45-0.52&Maximum water penetration\\
        2&Visible green&0.52-0.60&Good for measuring plant vigor\\
        3&Visible red&0.63-0.69&Vegetation discrimination\\
        4&Near infrared&0.76-0.90&Biomass and shoreline mapping\\
        5&Middle infrared&1.55-1.75&Moisture content of soil and vegetation\\
        6&Thermal infrared&10.4-12.5&Soil moisture; thermal mapping\\
        7&Middle infrared&2.08-2.35&Mineral mapping\\
        \hline
    \end{tabular}
    \caption{Thematic bands in NASA's LANDSAT satellite.}
    \label{Table:1.1}
\end{table}

Table \ref{Table:1.1} shows the so-called \textit{thematic bands} in NASA's LANDSAT satellite.

In order to develop a basic appreciation for the power of this type of \textit{multispectral} imaging, consider Fig.\;\ref{Figure:1.10}, which shows one image for each of the spectral bands in Table \ref{Table:1.1}.

\begin{figure}[htbp]
    \centering
    \includegraphics[width=\linewidth]{Figure1 10.png}
    \caption{LANDSAT satellite images of the Washionton, D.C. area. The number refer to the thematic bands in Table \ref{Table:1.1}. (Images courtesy of NASA).}
    \label{Figure:1.10}
\end{figure}

Figures \ref{Figure:1.12} and \ref{Figure:1.13} show an application of infrared imaging. These images are part of the \textit{Nighttime Lights of the World} data set, which provides a global inventory of human settlements.

\begin{figure}[htbp]
    \centering
    \includegraphics[width=\linewidth]{Figure1 12.jpeg}
    \caption{Infrared satellite images of the Americas. The small grey map is provided for reference. (Courtesy of NOAA.)}
    \label{Figure:1.12}
\end{figure}

\begin{figure}[htbp]
    \centering
    \includegraphics[width=\linewidth]{Figure1 13.jpeg}
    \caption{Infrared satellite images of the remaining populated part of the world. The small grey map is provided for reference. (Courtesy of NOAA.)}
    \label{Figure:1.13}
\end{figure}

\subsection{Examples in which Other Imaging Modalities Are Used}

A \textit{transmission electron microscope} (TEM) works much like a slide projector. A \textit{scanning electron microscope} (SEM), on the other hand, actually scans the electron beam and records the interaction of beam and sample at each location.

\textit{Fractals} are striking examples of computer-generated images. Basically, a fractal is nothing more than an iterative reproduction of a basic pattern according to mathematical rules. For instance, \textit{tiling} is one of the simplest ways to generate a fractal image.

\section{Fundamental Steps in Digital Image Processing}

% %--------------------------------------------------------------------------
% %         Bibliographie 
% %--------------------------------------------------------------------------
\end{document}
